%% Пакеты для русского языка
%----------------------------------------------
\usepackage{cmap}                   % поиск в PDF
\usepackage[T2A]{fontenc}
\usepackage[utf8]{inputenc}
\usepackage[english,russian]{babel}


%% Работа с математикой
%------------------------------------------------------------------------------
\usepackage{amsmath, amsfonts,amssymb,amsthm,mathtools}
\usepackage{icomma}  % "Умная" запятая: --- $0,2$ - число, $0, 2$ - перечисление
\usepackage{blkarray}

% Эти пакеты содержат символы, отсутствующие при использовании пакета mathtext
% Совет взят из https://tex.stackexchange.com/a/355726
\usepackage{tempora}  % Times for numbers in math mode
\usepackage{newtxmath}  % Times in math mode

\usepackage{bm}  % жирные векторы

% Русские буквы в фомулах. Можно использовать вместо tempora и newtxmath
% \usepackage{mathtext}



% Показывать номера только у тех формул, на которые есть \eqref{} в тексте.
\mathtoolsset{showonlyrefs=true}
%\usepackage{leqno} % Немуреация формул слева

% Математические операторы:
\DeclareMathOperator{\sgn}{\mathop{sgn}}
\newcommand{\defeq}{\stackrel{\textrm{\scriptsize def}}{=}}
\DeclareMathOperator*{\argmax}{argmax}
\DeclareMathOperator*{\argmin}{argmin}


%% Дополнительные пакеты
%------------------------------------------------------------------------------
\usepackage[dvipsnames]{xcolor}
\usepackage[normalem]{ulem}   % перечеркивание текста с помощью \sout

\usepackage{comment}    % Позволяет убирать блоки текста (добавляет
                        % окружение comment и команду \excludecomment)
\usepackage{tcolorbox}

% Определяем окружения для комментирований и заметок
\ifnumequal{\value{review}}{1}{
    \usepackage[nowatermark]{fixmetodonotes}
    \setlength {\marginparwidth }{2cm}
    \renewcommand{\listnotesname}{\ListOfTodosTitle}
    \newrobustcmd{\note}[2][]{\ifstrempty{#1}{\colorbox{yellow}{\textit{#2}}}{\colorbox{#1}{\textit{#2}}}}
    \newenvironment{commentbox}
    {\begin{center}\begin{tcolorbox}[colback=yellow!20!white,colframe=yellow!80!white]\begin{em}}
    {\end{em}\end{tcolorbox}\end{center}}
}{
    \usepackage[disabled]{fixmetodonotes}
    \renewcommand{\FIXME}[1]{#1}
    \newrobustcmd{\note}[2][]{}
    \excludecomment{commentbox}
}


%%% Нумерация русскими буквами
%%%   пример:
%%%     \begin{enumerate}[label=\asbuk*),ref=\asbuk*]
%%%       \item one
%%%       \item two
%%%     \end{enumerate}
\usepackage{enumitem}
\makeatletter
\AddEnumerateCounter{\asbuk}{\russian@alph}{щ}
\makeatother

\usepackage{color}   % Пакет цветов, для раскраски ссылок
\usepackage{hyperref}  %

\hypersetup{
    colorlinks=true, % set true if you want colored links
    linktoc=all,     % set to all if you want both sections and subsections linked
    linkcolor=blue,  % choose some color if you want links to stand out
}

\usepackage{lipsum}  % позволяет вставлять lorem ipsum


%% Работа с графикой
%------------------------------------------------------------------------------
\usepackage{graphicx}
\graphicspath{{img/}}
\setlength\fboxsep{3pt}   % Отступ рамки \fbox{} от рисунка
\setlength\fboxrule{1pt}  % Толщина линий рамки \fbox{}
\usepackage{wrapfig}      % Обтекание рисунков текстом
\usepackage{float}
\usepackage{caption}
\usepackage{subcaption}   % Для вставки подписей к нескольким картинкам в одной



%% Работа с таблицами
%------------------------------------------------------------------------------
\usepackage{array,tabularx,tabulary,booktabs}
\usepackage{longtable}  % Длинные таблицы
\usepackage{multirow}   % Слияние строк в таблице


% Теоремы
%------------------------------------------------------------------------------
\theoremstyle{plain} % Это стиль по умолчанию, его можно не переопределять.
\newtheorem{theorem}{Теорема}[section]
\newtheorem{proposition}{Утверждение}[section]
\newtheorem{lemma}{Лемма}[section]

\theoremstyle{definition} % "Определение"
\newtheorem{corollary}{Следствие}[theorem]
\newtheorem{propcorollary}{Следствие}[proposition]
\newtheorem{problem}{Задача}[section]
\newtheorem{definition}{Определение}[section]

\theoremstyle{remark} % "Примечание"
\newtheorem{remark}{Замечание}[section]
\newtheorem*{nonum}{Решение}
% TODO: посмотреть, как в диссертации сделаны теоремы и доказательства


%% Программирование
%------------------------------------------------------------------------------
\usepackage{etoolbox}  % логические операторы
\usepackage{listings}


%% Библиограция
\usepackage[
    backend=biber,      % движок
    bibencoding=utf8,   % кодировка bib файла
    sorting=none,       % настройка сортировки списка литературы
    style=gost-numeric, % стиль цитирования и библиографии (по ГОСТ)
    language=autobib,   % получение языка из babel/polyglossia, default: autobib
                        % если ставить autocite или auto, то цитаты в тексте с
                        % указанием  страницы, получат указание страницы на
                        % языке оригинала
    autolang=other,     % многоязычная библиография
    clearlang=true,     % внутренний сброс поля language, если он совпадает с
                        % языком из babel/polyglossia
    defernumbers=true,  % нумерация проставляется после двух компиляций, зато
                        % позволяет выцеплять библиографию по ключевым словам и
                        % нумеровать не из большего списка
    sortcites=true,     % сортировать номера затекстовых ссылок при цитировании
                        % (если в квадратных скобках несколько ссылок, то
                        % отображаться будут   отсортированно, а не абы как)
    doi=true,           % Показывать или нет ссылки на DOI
    url=true,           % Показывать или нет URL
    isbn=false          % Показывать или нет ISBN, ISSN, ISRN
]{biblatex}

\usepackage{csquotes}
\addbibresource{observe.bib}
